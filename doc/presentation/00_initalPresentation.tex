%%%
%%% Copyright (c) 2010 Jens Haase <je.haase@googlemail.com>
%%%
%%% Permission is hereby granted, free of charge, to any person obtaining a copy
%%% of this software and associated documentation files (the "Software"), to deal
%%% in the Software without restriction, including without limitation the rights
%%% to use, copy, modify, merge, publish, distribute, sublicense, and/or sell
%%% copies of the Software, and to permit persons to whom the Software is
%%% furnished to do so, subject to the following conditions:
%%%
%%% The above copyright notice and this permission notice shall be included in
%%% all copies or substantial portions of the Software.
%%%
%%% THE SOFTWARE IS PROVIDED "AS IS", WITHOUT WARRANTY OF ANY KIND, EXPRESS OR
%%% IMPLIED, INCLUDING BUT NOT LIMITED TO THE WARRANTIES OF MERCHANTABILITY,
%%% FITNESS FOR A PARTICULAR PURPOSE AND NONINFRINGEMENT. IN NO EVENT SHALL THE
%%% AUTHORS OR COPYRIGHT HOLDERS BE LIABLE FOR ANY CLAIM, DAMAGES OR OTHER
%%% LIABILITY, WHETHER IN AN ACTION OF CONTRACT, TORT OR OTHERWISE, ARISING FROM,
%%% OUT OF OR IN CONNECTION WITH THE SOFTWARE OR THE USE OR OTHER DEALINGS IN
%%% THE SOFTWARE.
%%%

\documentclass[accentcolor=tud9b, colorbacktitle, inverttitle]{tudbeamer}

\usepackage[utf8]{inputenc}
\usepackage[english]{babel}
% \usepackage{fontenc}
% \usepackage{graphicx}
\usepackage{url}
% \usepackage{cite}
% \usepackage{longtable}
% \usepackage{listings}
% \usepackage{wrapfig}
% \usepackage[numbers]{natbib}

\begin{document}

\author{Jens Haase} \title{Analyzing German Noun Compounds using a
  Web-Scale Dataset -- Task description} \subtitle{UIMA Software
  Project WS 2010/2011}

\begin{titleframe}
\end{titleframe}

\begin{frame}
  \frametitle{Introduction}
  \begin{itemize}
  \item Noun-compounding: Combination of two existing words to another
    new word.
  \item Powerful feature in the German language
  \item Example: \emph{Blumensträuße} (flower bouquet) ->
    \emph{Blumen} (flower) + \emph{Sträuße} (bouquet)
  \end{itemize}

  Problem in many NLP task
  \begin{itemize}
  \item Search for a compound word should also include result with the
    words splitted
  \item Example: \emph{Lackschicht} (paint layer) should return
    results with the words \emph{Lackschicht} and \emph{Schicht aus
      Lack} (layer of paint)
  \end{itemize}
\end{frame}

\begin{frame}
  \frametitle{Problem definition}
  \begin{itemize}
  \item Compounds are formed with nouns, verbs and adjectives.
  \item Compound words can be compound with other
  \item Linking morphemes are added between words:
    \emph{Tag(es)+ration}
  \item Different context for different splits: \emph{Tag(es)+ration}
    vs. \emph{Tag(es)+rat+ion}
  \end{itemize}


  Main algorithm \cite{alf2008}
  \begin{enumerate}
  \item Calculate every possible way of splitting a word in one or
    more parts
  \item Score those parts according to some weighting function
  \item Take the highest-scoring decomposition. If it contains one
    part, it means that the word in not a compound.
  \end{enumerate}
\end{frame}

\begin{frame}
  \frametitle{Roadmap}

  \begin{tabular}{|l|l|}
  \hline
  \textbf{Week} & \textbf{Goals} \\ \hline
  08.11 - 14.11. & get familiar with the project; choosing dictionary \\ \hline
  15.11 - 21.11. & access to dictionary \\ \hline
  22.11 - 28.11. & access Google Web1T; splitting words \\ \hline
  29.11 - 05.06. & splitting words \\ \hline
  06.12 - 12.12. & splitting words \\ \hline
  13.12 - 19.12. & evaluation and testing \\ \hline
  20.12 - 26.12. & weighting function (Christmas) \\ \hline
  27.12 - 02.01. & weighting function (Christmas, new years eve) \\ \hline
  03.01 - 09.01. & weighting function \\ \hline
  10.01 - 16.01. & no time (vacation) \\ \hline
  17.01 - 23.01. & evaluation and testing \\ \hline
  24.01 - 30.01. & UIMA Component \\ \hline
  31.01 - 06.02. & project cleanup \\
  \hline
  \end{tabular}
\end{frame}

\begin{frame}
  \frametitle{End}
  
\begin{block}{Questions}
  Ask now, or later.
\end{block}

\begin{block}{More information}
  Code, documentation and slides are available on github: \url{https://github.com/jenshaase/noun-decompounds}
\end{block}
\end{frame}

\begin{frame}
  \frametitle{References}
  
  \bibliographystyle{alpha}
  \bibliography{../documents/00_taskdescription}

\end{frame}

\end{document}
