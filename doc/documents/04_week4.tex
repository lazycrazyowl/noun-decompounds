%%%
%%% Copyright (c) 2010 Jens Haase <je.haase@googlemail.com>
%%%
%%% Permission is hereby granted, free of charge, to any person obtaining a copy
%%% of this software and associated documentation files (the "Software"), to deal
%%% in the Software without restriction, including without limitation the rights
%%% to use, copy, modify, merge, publish, distribute, sublicense, and/or sell
%%% copies of the Software, and to permit persons to whom the Software is
%%% furnished to do so, subject to the following conditions:
%%%
%%% The above copyright notice and this permission notice shall be included in
%%% all copies or substantial portions of the Software.
%%%
%%% THE SOFTWARE IS PROVIDED "AS IS", WITHOUT WARRANTY OF ANY KIND, EXPRESS OR
%%% IMPLIED, INCLUDING BUT NOT LIMITED TO THE WARRANTIES OF MERCHANTABILITY,
%%% FITNESS FOR A PARTICULAR PURPOSE AND NONINFRINGEMENT. IN NO EVENT SHALL THE
%%% AUTHORS OR COPYRIGHT HOLDERS BE LIABLE FOR ANY CLAIM, DAMAGES OR OTHER
%%% LIABILITY, WHETHER IN AN ACTION OF CONTRACT, TORT OR OTHERWISE, ARISING FROM,
%%% OUT OF OR IN CONNECTION WITH THE SOFTWARE OR THE USE OR OTHER DEALINGS IN
%%% THE SOFTWARE.
%%%

\documentclass[11pt, accentcolor=tud9b, nochapname]{tudexercise}

\usepackage[utf8]{inputenc}
\usepackage[english]{babel}
% \usepackage{fontenc}
% \usepackage{graphicx}
\usepackage{url}
% \usepackage{cite}
\usepackage{longtable}
\usepackage{listings}
% \usepackage{wrapfig}
% \usepackage[numbers]{natbib}

\begin{document}

\author{Jens Haase}
\title{Analyzing German Noun Compounds using a
  Web-Scale Dataset -- Report Week 4}
\subtitle{UIMA Software Project WS 2010/2011}
\subsubtitle{Jens Haase}
\date{\today}
\maketitle

\section{Work done in the last week}

\subsection{Refactoring and Improving left to right algorithm}
The first step of this week was to refactor the left to right algorithm. The goal was that the algorithm should return a tree instead of list. The tree structure has the benefit that it can be visualized. The visualization of the splitting algorithm can be used to see how the algorithm works and which errors occur. The new structure can later also be used for the ranking algorithm.

The old algorithm was moved to a separated package and marked as deprecated. It is only available to compared the result with the new one.

\subsection{New Data driven algorithm}
Next to the \emph{left to right algorithm} there is now a new splitting algorithm. This splitting algorithm tries to split the word based on the amount of words that begin and end with the same letters as the word that should be splitted. A detailed description can be found at \url{http://citeseerx.ist.psu.edu/viewdoc/download?doi=10.1.1.35.7240&rep=rep1&type=pdf}.

For the algorithm a trie implementation is needed. This implementation can be found in the \texttt{trie} package. The trie was only tested with the IGerman98 dictionary. Dictionaries with a lot of more words can be ran out of memory. For that a trie with a higher performance is required.

The implementation of the algorithm itself also returns a tree, as mention above in the new left to right algorithm.

\subsection{Current Evaluation}
Due to a bug in the evaluation class, the results of the last week are not correct. See the following table for the current evaluation of the algorithm:

\vspace{10pt}
\begin{tabular}{l | l | l | l }
  \hline
  \textbf{ } & \textbf{Left-to-Right (Old)} & \textbf{Left-to-Right (New)} & \textbf{Data-Driven-Algorithm} \\ \hline
  \textbf{Recall with morphemes} & 0.44 & 0.47 & 0.16 \\ \hline
  \textbf{Recall without morphemes} & 0.83 & 0.88 & 0.41 \\ \hline
\end{tabular}
\vspace{10pt}

As we can see the new \emph{left to right algorithm} works best. But it requires some future work to find the right morphemes. The splits are in most cases a the right position.

The new data driven algorithm returns very bad results. The recall without morphemes is more worse the the recall with morphemes of the \emph{left to right algorithm}. I think improving this algorithm will never beat the \emph{left to right algorithm}.

\subsection{Time Tracking}

\begin{tabular}{l | l | l | l}
  \hline
  \textbf{Date} & \textbf{Task} & \textbf{Needed time} & \textbf{Planned time} \\ \hline
  2010-12-02 & New Data Driven Algorithm (Trie) & 4 & 4 \\ \hline
  2010-12-06 & New Data Driven Algorithm & 3 & 4 \\ \hline
  2010-12-07 & Refactor/Improve left to right algorithm & 5 & 5 \\ \hline
  2010-12-08 & Report writing & 2 & 2  \\ \hline
\end{tabular}

\section{Plan for next week}
The focus for the next week is to improve the new left to right algorithm. Especially, the recall with morphemes should be higher. In the next week I also want to start to think about a ranking function. This should result in different concepts that can be implemented in the following weeks.

\vspace{10pt}
\begin{tabular}{l | l | l}
  \hline
  \textbf{Date} & \textbf{Task} & \textbf{Planned time} \\ \hline
  2010-12-09 & Improve splitting algorithm & 4 \\ \hline
  2010-12-12 & Ranking functions & 3 \\ \hline
  2010-12-13 & Ranking functions & 2 \\ \hline
  2010-12-15 & Report Writing & 2 \\ \hline
\end{tabular}

\end{document}
