%%%
%%% Copyright (c) 2010 Jens Haase <je.haase@googlemail.com>
%%%
%%% Permission is hereby granted, free of charge, to any person obtaining a copy
%%% of this software and associated documentation files (the "Software"), to deal
%%% in the Software without restriction, including without limitation the rights
%%% to use, copy, modify, merge, publish, distribute, sublicense, and/or sell
%%% copies of the Software, and to permit persons to whom the Software is
%%% furnished to do so, subject to the following conditions:
%%%
%%% The above copyright notice and this permission notice shall be included in
%%% all copies or substantial portions of the Software.
%%%
%%% THE SOFTWARE IS PROVIDED "AS IS", WITHOUT WARRANTY OF ANY KIND, EXPRESS OR
%%% IMPLIED, INCLUDING BUT NOT LIMITED TO THE WARRANTIES OF MERCHANTABILITY,
%%% FITNESS FOR A PARTICULAR PURPOSE AND NONINFRINGEMENT. IN NO EVENT SHALL THE
%%% AUTHORS OR COPYRIGHT HOLDERS BE LIABLE FOR ANY CLAIM, DAMAGES OR OTHER
%%% LIABILITY, WHETHER IN AN ACTION OF CONTRACT, TORT OR OTHERWISE, ARISING FROM,
%%% OUT OF OR IN CONNECTION WITH THE SOFTWARE OR THE USE OR OTHER DEALINGS IN
%%% THE SOFTWARE.
%%%

\documentclass[11pt, accentcolor=tud9b, nochapname]{tudexercise}

\usepackage[utf8]{inputenc}
\usepackage[english]{babel}
% \usepackage{fontenc}
% \usepackage{graphicx}
\usepackage{url}
% \usepackage{cite}
\usepackage{longtable}
\usepackage{listings}
% \usepackage{wrapfig}
% \usepackage[numbers]{natbib}

\begin{document}

\author{Jens Haase}
\title{Analyzing German Noun Compounds using a
  Web-Scale Dataset -- Report Week 3}
\subtitle{UIMA Software Project WS 2010/2011}
\subsubtitle{Jens Haase}
\date{\today}
\maketitle

\section{Work done in the last week}

\subsection{Improving dictionary}
The first thing of this week was to improve the igerman98
dictionary. For that the flag of each word is handled by the dictionary
reader. The flag identify rules. These rules come in a seperated text
file with the ending \texttt{aff} for affix. The rules mostly represent prefixes and suffixes, which change the word at the beginning or the end. All flags can
also be combined. But this is currently not implemented.

The \texttt{Finder} class can now also be used as dictionary. It implements
now the \texttt{IDictionary} interface. Using this
dictionary with the current implementaion of the splitting algorithm
is very unperformand. I tested the performace with a the
\texttt{FinderPerformaceTest} class in the \texttt{test}
dictionary. Most of the operation on the index are very fast. But
sometimes it takes up to two seconds. Running the evaluation takes
very long because the current splitting algorithm makes heavy use of
the dictionary. After around 50 words of the corpus and 1 hour of waiting
I stopped the program.

The first 50 result also showed me that the unigrams of the web1t
corpus do not have a high quality. The corpus contains a lot of words
that are not real words.

\subsection{Improving left to right algorithm}

With the improved dictionary I hope that also the recall of around
0.45 will be better. But in the first run the recall becomes more
worse. It only was by around 0.43. Looking at the result, a lot words look very good, but never exactly like the correct one. The most
error cases are the morphemes (position of brackets in textual representation).

In the next evaluation steps I ignored morphemes and only looked if
the split was a the right position. That results in a recall of 0.826,
which is very good compared to 0.43. This indicates that the real
problem is not the split position. It is a problem of how to set the
morphemes. Imagine the correct form of the word \emph{Aktionsplan} is
\emph{Akt+ions+plan}, but the current algorithm only returns
\emph{Akt+ion(s)+plan}. For this example it could be a good idea to iterate over
all words and check if the word combined with the morpheme is also a
valid word (e.g. check for \emph{ions}). All combination of these new words are candidates for the right split. For evaluation of the improvment see the next section.

\subsection{Current Evaluation}
With the current status of the algorithm the recall of correct splits
is by 0.44. Without looking at morphemes we have a recall of 0.96. This
mean 96\% of all word are splitted at the right position. The words
only differ in the position of morphemes.


\subsection{Time Tracking}

\begin{tabular}{l | l | l | l}
  \hline
  \textbf{Date} & \textbf{Task} & \textbf{Needed time} & \textbf{Planned time} \\ \hline
  2010-11-25 & Better dictionary reader & 3 & 4 \\ \hline
  2010-11-27 & Changed Evaluation (improve Splitter) & 2 & 3 \\ \hline
  2010-11-30 & Used Finder as dictionary (improve Splitter) & 2 & 2 \\ \hline
  2010-12-01 & Improving left-to-right algorithm, Report writing & 6 & 4+2  \\ \hline
\end{tabular}

\section{Plan for next week}
In the next week I will go more in detail with the morphemes and try
to improve the recall value. I also want to try a new data driven
algorithm as mentioned by Larson et
al. \url{http://citeseerx.ist.psu.edu/viewdoc/download?doi=10.1.1.35.7240\&rep=rep1\&type=pdf}. At the end I will try to compare the two algorithm.


\vspace{10pt}
\begin{tabular}{l | l | l}
  \hline
  \textbf{Date} & \textbf{Task} & \textbf{Planned time} \\ \hline
  2010-12-02 & Improve splitting algorithm & 5 \\ \hline
  2010-12-04 & New data driven algorithm & 2 \\ \hline
  2010-12-06 & New data driven algorithm & 2 \\ \hline
  2010-12-09 & New data driven algorithm, Evaluation, Report Writing & 6 \\ \hline
\end{tabular}

\end{document}