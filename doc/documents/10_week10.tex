%%%
%%% Copyright (c) 2010 Jens Haase <je.haase@googlemail.com>
%%%
%%% Permission is hereby granted, free of charge, to any person obtaining a copy
%%% of this software and associated documentation files (the "Software"), to deal
%%% in the Software without restriction, including without limitation the rights
%%% to use, copy, modify, merge, publish, distribute, sublicense, and/or sell
%%% copies of the Software, and to permit persons to whom the Software is
%%% furnished to do so, subject to the following conditions:
%%%
%%% The above copyright notice and this permission notice shall be included in
%%% all copies or substantial portions of the Software.
%%%
%%% THE SOFTWARE IS PROVIDED "AS IS", WITHOUT WARRANTY OF ANY KIND, EXPRESS OR
%%% IMPLIED, INCLUDING BUT NOT LIMITED TO THE WARRANTIES OF MERCHANTABILITY,
%%% FITNESS FOR A PARTICULAR PURPOSE AND NONINFRINGEMENT. IN NO EVENT SHALL THE
%%% AUTHORS OR COPYRIGHT HOLDERS BE LIABLE FOR ANY CLAIM, DAMAGES OR OTHER
%%% LIABILITY, WHETHER IN AN ACTION OF CONTRACT, TORT OR OTHERWISE, ARISING FROM,
%%% OUT OF OR IN CONNECTION WITH THE SOFTWARE OR THE USE OR OTHER DEALINGS IN
%%% THE SOFTWARE.
%%%

\documentclass[11pt, accentcolor=tud9b, nochapname]{tudexercise}

\usepackage[utf8]{inputenc}
\usepackage[english]{babel}
% \usepackage{fontenc}
% \usepackage{graphicx}
\usepackage{url}
\usepackage{cite}
\usepackage{longtable}
\usepackage{listings}
% \usepackage{wrapfig}
% \usepackage[numbers]{natbib}

\begin{document}

\author{Jens Haase}
\title{Analyzing German Noun Compounds using a
  Web-Scale Dataset -- Report Week 10}
\subtitle{UIMA Software Project WS 2010/2011}
\subsubtitle{Jens Haase}
\date{\today}
\maketitle

\section{Work done in the last week}

\subsection{UIMA component}
To use the generated code of the last weeks in real applications a UIMA component was developed. This component can later be used in different projects. It consist of two annotator and one Annotation Type. The two components are nearly identically. They require a index path and a token class as parameter. As optional parameter the ranking algorithm can be set (default is the frequency algorithm). The difference of the two components is that one uses the ranking algorithms on a tree and the other on a list, as we saw last week.

Before using the component in an analysis engine a tokenizer is required. The component iterates over the all token and tries to split each one. As result a set of SplittedToken Annotation will be added to the CAS. Each Annotation contains a StringArray with the list of individual words. In each individual word morphemes are marked with brackets. For example the token ``Aktionsplan'' is splitted in ``Aktion(s)+plan'' than the resulting string array is [``aktion(s)'', ``plan''].

\subsection{Code cleanup and Refactoring}
For the final delivery a lot of code cleanup and refactoring was done. Javadoc was added to all classes and command line tools are created.

\subsection{Evaluation on a larger dataset}
At least the evaluation ran on a larger dataset. Currently 10,000 entries were evaluated. A evaluation of the the complete dataset takes several days or requires a faster machine. Here are the results:

\vspace{10pt}
\begin{tabular}{l | l | l | l}
  \hline
  \textbf{Algorithm} & \textbf{Recall tree} & \textbf{Recall list} \\ \hline
  Frequency Based & TBD () & 0.5078 (0.7029) \\ \hline
  Probability Based & TBD () & 0.1843 (0.2558) \\ \hline
  Mutual Information Based & TBD () & TBD () \\ \hline
\end{tabular}
\vspace{10pt}


\subsection{Time Tracking}

\begin{tabular}{l | l | l | l}
  \hline
  \textbf{Date} & \textbf{Task} & \textbf{Needed time} & \textbf{Planned time} \\ \hline
  2011-01-30 & Javadoc & 2 & 2 \\ \hline
  2011-01-31 & Refactoring and Cleanup & 6 & 8 \\ \hline
  2011-02-02 & Uima Component & 4 & 4 \\ \hline
  2011-02-02 & Report Writing & 2 & 3 \\ \hline
\end{tabular}

\end{document}
