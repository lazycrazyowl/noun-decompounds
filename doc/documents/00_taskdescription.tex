\documentclass[11pt, accentcolor=tud9b, nochapname]{tudexercise}

\usepackage[utf8]{inputenc}
\usepackage[english]{babel}
% \usepackage{fontenc}
% \usepackage{graphicx}
\usepackage{url}
% \usepackage{cite}
\usepackage{longtable}
% \usepackage{listings}
% \usepackage{wrapfig}
% \usepackage[numbers]{natbib}

\begin{document}

\author{Jens Haase}
\title{Analyzing German Noun Compounds using a Web-Scale Dataset -- Task description}
\subtitle{UIMA Software Project WS 2010/2011}
\subsubtitle{Jens Haase}
\date{\today}
\maketitle

\section{Introduction}
Noun-compounding in the German language is a powerful process to build
new words with the combination of two or more existing
words. \emph{Blumensträuße} (flower bouquet) for example contains the
two words \emph{Blumen} (flower) and \emph{Sträuße} (bouquet).

In Unstructured Information Management this can be a handicap for
tools like Machine Translation, Speech Recognition or Information
Retrieval. Imagine you search for the word \emph{Lackschicht} (paint
layer), the expected result will be all documents containing the word
\emph{Lackschicht} but also the words \emph{Schicht aus Lack} (layer of
paint). To find a expected result the search engine must split all
compounds nouns in single words (\emph{Lackschicht} -> \emph{Lack schicht}).

\section{Problem definition}
In the German language a compounds can be formed by combining nouns,
verbs and adjectives. Also compound words can combined to a new
compound word. The word \emph{Donaudampfschifffahrtskapitänsmütze} (a
cap for captains of steamboats driven on the river Donau), contains the
words \emph{Donaudampfschiff} (steamboat on the river Donau) and
\emph{Kapitänsmütze} (the cap of the captain). Each word can be
split again. The final decompounds words are \emph{Donau},
\emph{Dampf}, \emph{Schiff}, \emph{Fahrt}, \emph{Kapitän} and
\emph{Mütze}.

Another problem in noun decompounds are linking morphemes. These
morphemes are allowed between two words. The linking morpheme in
\emph{Kaptiänsmütze} is the the \emph{s} between \emph{Kapitän} and
\emph{Mütze} (\emph{Kapitän(s)+mütze}). Also the word
\emph{Orangensaft} (orange juice) has a \emph{n} as linking morpheme
(\emph{Orange(n)+saft)}. But in this case the word \emph{Orangen} is a
valid word in the German language. It is the plural of the word
\emph{Orange}. But the word \emph{Kapitäns} in previous example is not
a valid word.

At least not all possible splits have the same meaning. The word
\emph{Tagesration} (recommended daily amount), contains the word
\emph{Tag} (day), \emph{Rat} (advice) and \emph{Ion} (ion) but also
the word \emph{Ration} (ration). Splitting the the word in three
parts is wrong, because the chemical ion has nothing to do with the
word \emph{Tagesration}. The correct decompounding here will be
\emph{Tag(es)+ration}.

\section{Task}
Decompounding of a word is mostly done in three steps \cite{alf2008}:

\begin{enumerate}
\item[1.] Calculate every possible way of splitting a word in one or more
  parts
\item[2.] Score those parts according to some weighting function
\item[3.] Take the highest-scoring decomposition. If it contains one part,
  it means that the word in not a compound.
\end{enumerate}

At first we have to split a word in all possible parts. For that we
need a dictionary with all existing words. It is recommend to use a
special dictionary for the analyzing corpus. This can simple be done
in extracting all tokens of a document collection. A list of linking
morphemes is also needed.

In the next step we need a weighting function for each decompound
possibility. The can be for example a simple count frequency of every
word. The words with more frequency get a higher weight than other. At
the end we can rank the weights. The decomposition with the highest
weight wins.

At the end we want to have UIMA component that can be used in other
projects. It should able to word with special dictionaries and linking
morphemes. Optional different weighting function can be chosen.

\section{Tools}

\subsection{Dictionary}
As mentioned early it is recommend to build a dictionary for a special
corpus, but in this case we do not have a special corpus. Instead we
want to use a \emph{every day} dictionary, which are used for spell
correction. ASpell and ispell are open source spell checker with
dictionaries for nearly every language. Also hunspell is a modern
spell checker, used in OpenOffice, Mozilla Firefox and Google
Chrome. The dictionaries of aspell and hunspell are mostly the same.

\subsection{Google Web1T}
Google Web1T corpus contains over 130 billion tokens of the German
language, distributed over n-grams ($n \epsilon [1,5]$). The corpus
can be used for the weighting function. The unigrams can also be used
as dictionary.

To search and access the data I will use \emph{jWeb1T}
(\url{http://hlt.fbk.eu/en/technology/jWeb1t}). This is an open source
  tool for efficiently searching on the Google Web1T corpus.

\section{Roadmap}

\begin{longtable}{|l|l|}
\hline
\textbf{Week} & \textbf{Goals} \\ \hline
08.11 - 14.11 & get familiar with the project; choosing dictionary \\ \hline
15.11 - 21.11 & access to dictionary \\ \hline
22.11 - 28.11 & access Google Web1T; splitting words \\ \hline
29.11 - 05.06 & splitting words \\ \hline
06.12 - 12.12 & splitting words \\ \hline
13.12 - 19.12 & evaluation and testing \\ \hline
20.12 - 26.12 & weighting function (Christmas) \\ \hline
27.12 - 02.01 & weighting function (Christmas, new years eve) \\ \hline
03.01 - 09.01 & weighting function \\ \hline
10.01 - 16.01 & no time (vacation) \\ \hline
17.01 - 23.01 & evaluation and testing \\ \hline
24.01 - 30.01 & UIMA Component \\ \hline
31.01 - 06.02 & project cleanup \\
\hline
\end{longtable}

\subsection{Next weeks}
In the next week I write a Component that can access the
directory. The dictionary is either the hunspell dictionary or the
unigrams in the Web1T dataset. I also want to create a list of
compound words with the best split. This can help me to see a lot of
variants for the implementation. The list can also be used for testing
and evaluation.


\bibliographystyle{alpha}
\bibliography{00_taskdescription}

\end{document}
