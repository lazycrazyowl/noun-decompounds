%%%
%%% Copyright (c) 2010 Jens Haase <je.haase@googlemail.com>
%%%
%%% Permission is hereby granted, free of charge, to any person obtaining a copy
%%% of this software and associated documentation files (the "Software"), to deal
%%% in the Software without restriction, including without limitation the rights
%%% to use, copy, modify, merge, publish, distribute, sublicense, and/or sell
%%% copies of the Software, and to permit persons to whom the Software is
%%% furnished to do so, subject to the following conditions:
%%%
%%% The above copyright notice and this permission notice shall be included in
%%% all copies or substantial portions of the Software.
%%%
%%% THE SOFTWARE IS PROVIDED "AS IS", WITHOUT WARRANTY OF ANY KIND, EXPRESS OR
%%% IMPLIED, INCLUDING BUT NOT LIMITED TO THE WARRANTIES OF MERCHANTABILITY,
%%% FITNESS FOR A PARTICULAR PURPOSE AND NONINFRINGEMENT. IN NO EVENT SHALL THE
%%% AUTHORS OR COPYRIGHT HOLDERS BE LIABLE FOR ANY CLAIM, DAMAGES OR OTHER
%%% LIABILITY, WHETHER IN AN ACTION OF CONTRACT, TORT OR OTHERWISE, ARISING FROM,
%%% OUT OF OR IN CONNECTION WITH THE SOFTWARE OR THE USE OR OTHER DEALINGS IN
%%% THE SOFTWARE.
%%%

\documentclass[11pt, accentcolor=tud9b, nochapname]{tudexercise}

\usepackage[utf8]{inputenc}
\usepackage[english]{babel}
% \usepackage{fontenc}
% \usepackage{graphicx}
\usepackage{url}
% \usepackage{cite}
\usepackage{longtable}
\usepackage{listings}
% \usepackage{wrapfig}
% \usepackage[numbers]{natbib}

\begin{document}

\author{Jens Haase} \title{Analyzing German Noun Compounds using a
  Web-Scale Dataset -- Report Week 1} \subtitle{UIMA Software Project
  WS 2010/2011} \subsubtitle{Jens Haase} \date{\today}
\maketitle

\section{Work done in the last week}

\subsection{Dictionary}
To create all possible splitting of compound word, I first need a
dictionary with a list of words. Today all common spell checker, like
\texttt{hunspell} or \texttt{ispell} use the \texttt{igerman98} dictionary
\url{http://www.j3e.de/ispell/igerman98/index_en.html}. This
dictionary contains over 80,000 words. Each word in this dictionary
can also have optional flags. These flags can change the prefix or
suffix of a word. But for our purpose we only want to work with the
given list of words. More information on the file formats can be found
on \url{http://pwet.fr/man/linux/fichiers_speciaux/hunspell}.

The dictionary file can be found at \texttt{src/main/resources/de\_DE.dic}. Use
the class \texttt{IGerman98Dictionary} to read and work on this file.

For corpus based dictionaries the class \texttt{SimpleDictionary} can be
used. The class reads a file with a list of word. Each line must
contain one word.

\subsection{Access to Google Web1T}
Accessing the Google Web1T corpus was not as easy as I thought. The
corpus is splitted in unigrams, bigrams, trigrams, fourgrams and
fivegrams. Each of these package contains one or more single archives
that have to be extracted. The extracted files result in one text
file, containing a sorted list of n-grams. The following listing shows
the file format of one row.

\begin{lstlisting}
  token-1 <space> ... <space> token-n <tab> frequency
\end{lstlisting}

While the compressed size of the German corpus is 3 GB, the extracted
files have a size of 11,3 GB and contain following data:

\vspace{10pt}
\begin{tabular}{l | l}
  \hline
  Number of tokens: & 131,435,672,897 \\ \hline
  Number of sentences: & 15,715,470,319 \\ \hline
  Number of unigrams: & 15,133,396 \\ \hline
  Number of bigrams: & 88,668,144 \\ \hline
  Number of trigrams: & 154,226,178 \\ \hline
  Number of fourgrams: & 140,670,210 \\ \hline
  Number of fivegrams: & 100,542,274 \\ \hline
  Number of n-grams: & 499,240,202 \\ \hline
\end{tabular}
\vspace{10pt}

The first plan to access the data in Java was to use the jWeb1T tool
(\url{http://hlt.fbk.eu/en/node/81}). This tool can find n-grams in a
logarithmic time. But it can only find the frequency for a given
n-gram. For example if the n-gram 'hello world' is listed in the
corpus with the frequency 50, the jWeb1T tool will find the n-gram by
searching for 'hello world', but not by searching for 'hello' or
'world'.

For the weighting function I want to search for 'hello' and get all
n-grams containing the word 'hello'. When I search 'hello world' I
want to find the all n-grams with the words 'hello' and 'world'.

A second tool is the Java API for Web-1T Corpus of Digital Pebble
\url{http://www.digitalpebble.com/resources.html}. The tool is
currently not free available, but it is possible to contact Digital
Pebble. According to Julien Nioche of Digital Pebble the tool is
comparable to jWeb1T, but they are currently working on a new version
that will be freely available soon. Since I have not the time to wait
for this tool I have to create my own indexer.

I simply used lucene to index all n-grams. One n-gram is one document
in the lucene index containing the n-gram and the frequency. All data
is stored in the index so the extracted Google Web1T files are no
longer needed. Creating an index can be done with following command in
the projects folder

\begin{lstlisting}
  ./bin/web1TLuceneIndexer.sh \ 
    --web1t PATH/TO/FOLDER/WITH/ALL/EXTRACTED/N-GRAM/FILES \
    --outputPath PATH/TO/LUCENE/INDEX/FOLDER
\end{lstlisting}

But keep in mind that the task takes very long a writes a huge amount
of data to the hard disk. For me the task runs around 5 hours (313
minutes). The index has a size of 23,4 GB.

Using this tool for the English corpus will fail because Lucene can
only handle 2,147,483,647 (Integer.MAX\_VALUE) documents. In the
English corpus are 3,795,790,711 n-grams.

To access the generated index the class \texttt{Finder} can be used.

\section{Plan for next week}
In the next week I will focus my work on the splitting algorithm. The
splitting algorithm should split a word in all possible decompound
forms. To test the algorithm I will try to create a large list of
words, with all possible splittings.

For the split algorithm I also need a list of linking morphemes. These can either be hard coded in Java code or in a separated text file.

\end{document}
